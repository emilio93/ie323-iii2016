% Copyright 2017 Emilio Rojas
%
% Permission is hereby granted, free of charge, to any person obtaining a copy of
% this software and associated documentation files (the "Software"), to deal in
% the Software without restriction, including without limitation the rights to
% use, copy, modify, merge, publish, distribute, sublicense, and/or sell copies of
% the Software, and to permit persons to whom the Software is furnished to do so,
% subject to the following conditions:
%
% The above copyright notice and this permission notice shall be included in all
% copies or substantial portions of the Software.
%
% THE SOFTWARE IS PROVIDED "AS IS", WITHOUT WARRANTY OF ANY KIND, EXPRESS OR
% IMPLIED, INCLUDING BUT NOT LIMITED TO THE WARRANTIES OF MERCHANTABILITY, FITNESS
% FOR A PARTICULAR PURPOSE AND NONINFRINGEMENT. IN NO EVENT SHALL THE AUTHORS OR
% COPYRIGHT HOLDERS BE LIABLE FOR ANY CLAIM, DAMAGES OR OTHER LIABILITY, WHETHER
% IN AN ACTION OF CONTRACT, TORT OR OTHERWISE, ARISING FROM, OUT OF OR IN
% CONNECTION WITH THE SOFTWARE OR THE USE OR OTHER DEALINGS IN THE SOFTWARE.

\documentclass{article}

% Copyright 2017 Emilio Rojas
%
% Permission is hereby granted, free of charge, to any person obtaining a copy of
% this software and associated documentation files (the "Software"), to deal in
% the Software without restriction, including without limitation the rights to
% use, copy, modify, merge, publish, distribute, sublicense, and/or sell copies of
% the Software, and to permit persons to whom the Software is furnished to do so,
% subject to the following conditions:
%
% The above copyright notice and this permission notice shall be included in all
% copies or substantial portions of the Software.
%
% THE SOFTWARE IS PROVIDED "AS IS", WITHOUT WARRANTY OF ANY KIND, EXPRESS OR
% IMPLIED, INCLUDING BUT NOT LIMITED TO THE WARRANTIES OF MERCHANTABILITY, FITNESS
% FOR A PARTICULAR PURPOSE AND NONINFRINGEMENT. IN NO EVENT SHALL THE AUTHORS OR
% COPYRIGHT HOLDERS BE LIABLE FOR ANY CLAIM, DAMAGES OR OTHER LIABILITY, WHETHER
% IN AN ACTION OF CONTRACT, TORT OR OTHERWISE, ARISING FROM, OUT OF OR IN
% CONNECTION WITH THE SOFTWARE OR THE USE OR OTHER DEALINGS IN THE SOFTWARE.

% Estos son los paquetes que se cargan en todos los documentos, estos paquetes
% están en texlive-full (TODO separar paquetes por seccion de texlive donde
% está el paqeute). De momento se ordenan por longitud del nombre y
% alfabeticamente.

\usepackage{url}
\usepackage{cite}
\usepackage{tikz}

\usepackage{array}
\usepackage{color}
\usepackage{float}

\usepackage{empheq}

\usepackage{amsmath}
\usepackage{apacite}
\usepackage{caption}
\usepackage{gensymb}
\usepackage{siunitx}

\usepackage{amsfonts}
\usepackage{booktabs}
\usepackage{enumitem}
\usepackage{etoolbox}
\usepackage{fancyhdr}
\usepackage{graphicx}
\usepackage{geometry}
\usepackage{hyperref}
\usepackage{mdframed}
\usepackage{multirow}
\usepackage{pdfpages}
\usepackage{pgfplots}
\usepackage{setspace}
\usepackage{textcomp}

\usepackage{adjustbox}
\usepackage{mathtools}

\usepackage{circuitikz}
\usepackage{subcaption}

\usepackage[T1]{fontenc}
\usepackage[spanish]{babel}
\usepackage[utf8]{inputenc}

\input{./karnaugh.tex}
\input{../../shared/config.tex}
\input{../../shared/commands.tex}

\begin{document}

\pagenumbering{gobble}

\portada
{Tarea \#4}
{Emilio Javier Rojas Álvarez}
{B15680}
{\today}


Se construye el siguiente circuito:
\begin{figure}[H]
  \includegraphics[width=\textwidth]{circuito/main.png}
  \caption{Bloques del contador.}
\end{figure}
El bloque \textit{estado} es un banco de 6 flip flops, cada uno contiene el
estado presente de uno de los bits en el contador. A este le entra una señal de
reloj, una de set(pone todos los bits en 1), una de reset(pone todos los bits en
0), una para indicar si se puede o no alterar el valor que almacena el flip
flop, y la entrada de datos. El bloque \textit{6bit2bcd} convierte la señal de 6
bits a 14 salidas para 2 pantallas de 7 segmentos que representan el número de
la señal.

Se identifican mediante 2 bombillos si el contador está en 0 o el número máximo.
Se tienen dos bloques, \textit{-1} y \textit{+1} que suman o restan 1 al estado
presente del contador. Estas cajas se obtienen a partir de mapas de Karnaugh.

Mediante MUXES se selecciona la señal de 6 bits que indicará el próximo flanco
de reloj. Esta puede ser una de 3, una señal MNOPQR que se activa con LOAD, el
siguiente número, cuando MODE está activo y LOAD inactivo o bien el número
previo cuando MODE y LOAD estén inactivos.



\usection{Contador Ascendente}
\begin{table}[H]
  \centering
  \begin{adjustbox}{max width=\textwidth}
    \begin{tabular}{ c | c | c | c | c | c | c ||| c | c | c | c | c | c || c | c | c | c | c | c }
       & \multicolumn{6}{c}{\textbf{Número Presente}} &
       \multicolumn{6}{c}{\textbf{Número Siguiente} \small{$Q_5=0$}} &
       \multicolumn{6}{c}{\textbf{Número Siguiente} \small{$Q_5=1$}} \\
      \toprule
      Decimal  &
      $Q_5$ & $Q_4$ & $Q_3$ & $Q_2$ & $Q_1$ & $Q_0$ &
      $Q_5\prime$ & $Q_4\prime$ & $Q_3\prime$ & $Q_2\prime$ & $Q_1\prime$ & $Q_0\prime$ &
      $Q_5\prime$ & $Q_4\prime$ & $Q_3\prime$ & $Q_2\prime$ & $Q_1\prime$ & $Q_0\prime$ \\
      \toprule
      %       Q5    Q4  Q3  Q2  Q1  Q0    Q5  Q4  Q3  Q2  Q1  Q0    Q5  Q4  Q3  Q2  Q1  Q0
      0 /32 & 0/1 & 0 & 0 & 0 & 0 & 0 &   0 & 0 & 0 & 0 & 0 & 1 &   1 & 0 & 0 & 0 & 0 & 1 \\
      %       Q5    Q4  Q3  Q2  Q1  Q0    Q5  Q4  Q3  Q2  Q1  Q0    Q5  Q4  Q3  Q2  Q1  Q0
      1 /33 & 0/1 & 0 & 0 & 0 & 0 & 1 &   0 & 0 & 0 & 0 & 1 & 0 &   1 & 0 & 0 & 0 & 1 & 0 \\
      %       Q5    04  Q3  Q2  Q1  Q0    Q5  Q4  Q3  Q2  Q1  Q0    Q5  Q4  Q3  Q2  Q1  Q0
      2 /34 & 0/1 & 0 & 0 & 0 & 1 & 0 &   0 & 0 & 0 & 0 & 1 & 1 &   1 & 0 & 0 & 0 & 1 & 1 \\
      %       Q5    Q4  Q3  Q2  Q1  Q0    Q5  Q4  Q3  Q2  Q1  Q0    Q5  Q4  Q3  Q2  Q1  Q0
      3 /35 & 0/1 & 0 & 0 & 0 & 1 & 1 &   0 & 0 & 0 & 1 & 0 & 0 &   1 & 0 & 0 & 1 & 0 & 0 \\ \hline
      %       Q5    Q4  Q3  Q2  Q1  Q0    Q5  Q4  Q3  Q2  Q1  Q0    Q5  Q4  Q3  Q2  Q1  Q0
      4 /36 & 0/1 & 0 & 0 & 1 & 0 & 0 &   0 & 0 & 0 & 1 & 0 & 1 &   1 & 0 & 0 & 1 & 0 & 1 \\
      %       Q5    Q4  Q3  Q2  Q1  Q0    Q5  Q4  Q3  Q2  Q1  Q0    Q5  Q4  Q3  Q2  Q1  Q0
      5 /37 & 0/1 & 0 & 0 & 1 & 0 & 1 &   0 & 0 & 0 & 1 & 1 & 0 &   1 & 0 & 0 & 1 & 1 & 0 \\
      %       Q5    Q4  Q3  Q2  Q1  Q0    Q5  Q4  Q3  Q2  Q1  Q0    Q5  Q4  Q3  Q2  Q1  Q0
      6 /38 & 0/1 & 0 & 0 & 1 & 1 & 0 &   0 & 0 & 0 & 1 & 1 & 1 &   1 & 0 & 0 & 1 & 1 & 1 \\
      %       Q5    Q4  Q3  Q2  Q1  Q0    Q5  Q4  Q3  Q2  Q1  Q0    Q5  Q4  Q3  Q2  Q1  Q0
      7 /39 & 0/1 & 0 & 0 & 1 & 1 & 1 &   0 & 0 & 1 & 0 & 0 & 0 &   1 & 0 & 1 & 0 & 0 & 0 \\ \hline
      %       Q5    Q4  Q3  Q2  Q1  Q0    Q5  Q4  Q3  Q2  Q1  Q0    Q5  Q4  Q3  Q2  Q1  Q0
      8 /40 & 0/1 & 0 & 1 & 0 & 0 & 0 &   0 & 0 & 1 & 0 & 0 & 1 &   1 & 0 & 1 & 0 & 0 & 1 \\
      %       Q5    Q4  Q3  Q2  Q1  Q0    Q5  Q4  Q3  Q2  Q1  Q0    Q5  Q4  Q3  Q2  Q1  Q0
      9 /41 & 0/1 & 0 & 1 & 0 & 0 & 1 &   0 & 0 & 1 & 0 & 1 & 0 &   1 & 0 & 1 & 0 & 1 & 0 \\
      %       Q5    Q4  Q3  Q2  Q1  Q0    Q5  Q4  Q3  Q2  Q1  Q0    Q5  Q4  Q3  Q2  Q1  Q0
      10/42 & 0/1 & 0 & 1 & 0 & 1 & 0 &   0 & 0 & 1 & 0 & 1 & 1 &   1 & 0 & 1 & 0 & 1 & 1 \\
      %       Q5    Q4  Q3  Q2  Q1  Q0    Q5  Q4  Q3  Q2  Q1  Q0    Q5  Q4  Q3  Q2  Q1  Q0
      11/43 & 0/1 & 0 & 1 & 0 & 1 & 1 &   0 & 0 & 1 & 1 & 0 & 0 &   1 & 0 & 1 & 1 & 0 & 0 \\ \hline
      %       Q5    Q4  Q3  Q2  Q1  Q0    Q5  Q4  Q3  Q2  Q1  Q0    Q5  Q4  Q3  Q2  Q1  Q0
      12/44 & 0/1 & 0 & 1 & 1 & 0 & 0 &   0 & 0 & 1 & 1 & 0 & 1 &   1 & 0 & 1 & 1 & 0 & 1 \\
      %       Q5    Q4  Q3  Q2  Q1  Q0    Q5  Q4  Q3  Q2  Q1  Q0    Q5  Q4  Q3  Q2  Q1  Q0
      13/45 & 0/1 & 0 & 1 & 1 & 0 & 1 &   0 & 0 & 1 & 1 & 1 & 0 &   1 & 0 & 1 & 1 & 1 & 0 \\
      %       Q5    Q4  Q3  Q2  Q1  Q0    Q5  Q4  Q3  Q2  Q1  Q0    Q5  Q4  Q3  Q2  Q1  Q0
      14/46 & 0/1 & 0 & 1 & 1 & 1 & 0 &   0 & 0 & 1 & 1 & 1 & 1 &   1 & 0 & 1 & 1 & 1 & 1 \\
      %       Q5    Q4  Q3  Q2  Q1  Q0    Q5  Q4  Q3  Q2  Q1  Q0    Q5  Q4  Q3  Q2  Q1  Q0
      15/47 & 0/1 & 0 & 1 & 1 & 1 & 1 &   0 & 1 & 0 & 0 & 0 & 0 &   1 & 1 & 0 & 0 & 0 & 0 \\ \hline
      %       Q5    Q4  Q3  Q2  Q1  Q0    Q5  Q4  Q3  Q2  Q1  Q0    Q5  Q4  Q3  Q2  Q1  Q0
      16/48 & 0/1 & 1 & 0 & 0 & 0 & 0 &   0 & 1 & 0 & 0 & 0 & 1 &   1 & 1 & 0 & 0 & 0 & 1 \\
      %       Q5    Q4  Q3  Q2  Q1  Q0    Q5  Q4  Q3  Q2  Q1  Q0    Q5  Q4  Q3  Q2  Q1  Q0
      17/49 & 0/1 & 1 & 0 & 0 & 0 & 1 &   0 & 1 & 0 & 0 & 1 & 0 &   1 & 1 & 0 & 0 & 1 & 0 \\
      %       Q5    Q4  Q3  Q2  Q1  Q0    Q5  Q4  Q3  Q2  Q1  Q0    Q5  Q4  Q3  Q2  Q1  Q0
      18/50 & 0/1 & 1 & 0 & 0 & 1 & 0 &   0 & 1 & 0 & 0 & 1 & 1 &   1 & 1 & 0 & 0 & 1 & 1 \\
      %       Q5    Q4  Q3  Q2  Q1  Q0    Q5  Q4  Q3  Q2  Q1  Q0    Q5  Q4  Q3  Q2  Q1  Q0
      19/51 & 0/1 & 1 & 0 & 0 & 1 & 1 &   0 & 1 & 0 & 1 & 0 & 0 &   1 & 1 & 0 & 1 & 0 & 0 \\ \hline
      %       Q5    Q4  Q3  Q2  Q1  Q0    Q5  Q4  Q3  Q2  Q1  Q0    Q5  Q4  Q3  Q2  Q1  Q0
      20/52 & 0/1 & 1 & 0 & 1 & 0 & 0 &   0 & 1 & 0 & 1 & 0 & 1 &   1 & 1 & 0 & 1 & 0 & 1 \\
      %       Q5    Q4  Q3  Q2  Q1  Q0    Q5  Q4  Q3  Q2  Q1  Q0    Q5  Q4  Q3  Q2  Q1  Q0
      21/53 & 0/1 & 1 & 0 & 1 & 0 & 1 &   0 & 1 & 0 & 1 & 1 & 0 &   1 & 1 & 0 & 1 & 1 & 0 \\
      %       Q5    Q4  Q3  Q2  Q1  Q0    Q5  Q4  Q3  Q2  Q1  Q0    Q5  Q4  Q3  Q2  Q1  Q0
      22/54 & 0/1 & 1 & 0 & 1 & 1 & 0 &   0 & 1 & 0 & 1 & 1 & 1 &   1 & 1 & 0 & 1 & 1 & 1 \\
      %       Q5    Q4  Q3  Q2  Q1  Q0    Q5  Q4  Q3  Q2  Q1  Q0    Q5  Q4  Q3  Q2  Q1  Q0
      23/55 & 0/1 & 1 & 0 & 1 & 1 & 1 &   0 & 1 & 1 & 0 & 0 & 0 &   1 & 1 & 1 & 0 & 0 & 0 \\ \hline
      %       Q5    Q4  Q3  Q2  Q1  Q0    Q5  Q4  Q3  Q2  Q1  Q0    Q5  Q4  Q3  Q2  Q1  Q0
      24/56 & 0/1 & 1 & 1 & 0 & 0 & 0 &   0 & 1 & 1 & 0 & 0 & 1 &   1 & 1 & 1 & 0 & 0 & 1 \\
      %       Q5    Q4  Q3  Q2  Q1  Q0    Q5  Q4  Q3  Q2  Q1  Q0    Q5  Q4  Q3  Q2  Q1  Q0
      25/57 & 0/1 & 1 & 1 & 0 & 0 & 1 &   0 & 1 & 1 & 0 & 1 & 0 &   1 & 1 & 1 & 0 & 1 & 0 \\
      %       Q5    Q4  Q3  Q2  Q1  Q0    Q5  Q4  Q3  Q2  Q1  Q0    Q5  Q4  Q3  Q2  Q1  Q0
      26/58 & 0/1 & 1 & 1 & 0 & 1 & 0 &   0 & 1 & 1 & 0 & 1 & 1 &   1 & 1 & 1 & 0 & 1 & 1 \\
      %       Q5    Q4  Q3  Q2  Q1  Q0    Q5  Q4  Q3  Q2  Q1  Q0    Q5  Q4  Q3  Q2  Q1  Q0
      27/59 & 0/1 & 1 & 1 & 0 & 1 & 1 &   0 & 1 & 1 & 1 & 0 & 0 &   1 & 1 & 1 & 1 & 0 & 0 \\ \hline
      %       Q5    Q4  Q3  Q2  Q1  Q0    Q5  Q4  Q3  Q2  Q1  Q0    Q5  Q4  Q3  Q2  Q1  Q0
      28/60 & 0/1 & 1 & 1 & 1 & 0 & 0 &   0 & 1 & 1 & 1 & 0 & 1 &   1 & 1 & 1 & 1 & 0 & 1 \\
      %       Q5    Q4  Q3  Q2  Q1  Q0    Q5  Q4  Q3  Q2  Q1  Q0    Q5  Q4  Q3  Q2  Q1  Q0
      29/61 & 0/1 & 1 & 1 & 1 & 0 & 1 &   0 & 1 & 1 & 1 & 1 & 0 &   1 & 1 & 1 & 1 & 1 & 0 \\
      %       Q5    Q4  Q3  Q2  Q1  Q0    Q5  Q4  Q3  Q2  Q1  Q0    Q5  Q4  Q3  Q2  Q1  Q0
      30/62 & 0/1 & 1 & 1 & 1 & 1 & 0 &   0 & 1 & 1 & 1 & 1 & 1 &   1 & 1 & 1 & 1 & 1 & 1 \\
      %       Q5    Q4  Q3  Q2  Q1  Q0    Q5  Q4  Q3  Q2  Q1  Q0    Q5  Q4  Q3  Q2  Q1  Q0
      31/63 & 0/1 & 1 & 1 & 1 & 1 & 1 &   1 & 0 & 0 & 0 & 0 & 0 &   0 & 0 & 0 & 0 & 0 & 0 \\ \bottomrule
      %       Q5    Q4  Q3  Q2  Q1  Q0    Q5  Q4  Q3  Q2  Q1  Q0    Q5  Q4  Q3  Q2  Q1  Q0
    \end{tabular}
  \end{adjustbox}
  \caption{Tabla de verdad para el número siguiente de 6bits. Se muestra del 0 al
  31 en la columna \textbf{Número Siguiente} \small{$Q_5=0$}, del 32 al 63 en la
  columna \textbf{Número Siguiente} \small{$Q_5=1$}}
\end{table}

Se resuelven los mapas de Karnaugh para cada bit. $Q_5\prime$, $Q_4\prime$,
$Q_3\prime$, $Q_2\prime$, $Q_1\prime$ y $Q_0\prime$.
% Copyright 2017 Emilio Rojas
%
% Permission is hereby granted, free of charge, to any person obtaining a copy of
% this software and associated documentation files (the "Software"), to deal in
% the Software without restriction, including without limitation the rights to
% use, copy, modify, merge, publish, distribute, sublicense, and/or sell copies of
% the Software, and to permit persons to whom the Software is furnished to do so,
% subject to the following conditions:
%
% The above copyright notice and this permission notice shall be included in all
% copies or substantial portions of the Software.
%
% THE SOFTWARE IS PROVIDED "AS IS", WITHOUT WARRANTY OF ANY KIND, EXPRESS OR
% IMPLIED, INCLUDING BUT NOT LIMITED TO THE WARRANTIES OF MERCHANTABILITY, FITNESS
% FOR A PARTICULAR PURPOSE AND NONINFRINGEMENT. IN NO EVENT SHALL THE AUTHORS OR
% COPYRIGHT HOLDERS BE LIABLE FOR ANY CLAIM, DAMAGES OR OTHER LIABILITY, WHETHER
% IN AN ACTION OF CONTRACT, TORT OR OTHERWISE, ARISING FROM, OUT OF OR IN
% CONNECTION WITH THE SOFTWARE OR THE USE OR OTHER DEALINGS IN THE SOFTWARE.

\begin{figure}[H]
  \centering
  \caption{Mapa de Karnaugh para $Q_5\prime$ del contador ascendente.}

  \begin{subfigure}{0.1\textwidth}
    \centering
    \begin{tikzpicture}[scale=0.8]
      \draw(0,0) -- (0,0);
    \end{tikzpicture}
  \end{subfigure}
  \begin{subfigure}{.4\textwidth}
    \centering
    \begin{Karnaugh}{$Q_3$}{$Q_2$}{$Q_1$}{$Q_0$}
    \end{Karnaugh}
  \end{subfigure}
  \begin{subfigure}{.4\textwidth}
    \centering
    \begin{Karnaugh}{$Q_3$}{$Q_2$}{$Q_1$}{$Q_0$}
      \minterms{15}
      \implicantsol{15}{blue}
    \end{Karnaugh}
  \end{subfigure}

  \begin{subfigure}{0.1\textwidth}
    \centering
    \begin{tikzpicture}[scale=0.8]
      \draw[very thick] (0,0.3) -- node [left]{$Q_5$} ++(0,4);
    \end{tikzpicture}
  \end{subfigure}
  \begin{subfigure}{.4\textwidth}
    \centering
    \begin{Karnaugh}{$Q_3$}{$Q_2$}{$Q_1$}{$Q_0$}
      \minterms{0, 1, 2, 3, 4, 5, 6, 7, 8,  9, 10, 11, 12, 13, 14, 15}
      \implicant[-2pt]{0}{10}{red}
      \implicant[2pt]{0}{6}{green}
      \implicantdaltbaix[4pt]{0}{10}{orange}
      \implicant{0}{9}{cyan}
      \implicantcostats[2pt]{0}{10}{magenta}
    \end{Karnaugh}
  \end{subfigure}
  \begin{subfigure}{.4\textwidth}
    \centering
    \begin{Karnaugh}{$Q_3$}{$Q_2$}{$Q_1$}{$Q_0$}
      \minterms{0, 1, 2, 3, 4, 5, 6, 7, 8,  9, 10, 11, 12, 13, 14}
      \implicant[2pt]{0}{6}{green}
      \implicantdaltbaix[4pt]{0}{10}{orange}
      \implicant{0}{9}{cyan}
      \implicantcostats[2pt]{0}{10}{magenta}
    \end{Karnaugh}
  \end{subfigure}

  \begin{subfigure}{0.5\textwidth}
    \centering
    \begin{tikzpicture}[scale=0.8]
      \draw(0,0) -- (0,0);
    \end{tikzpicture}
  \end{subfigure}
  \begin{subfigure}{.4\textwidth}
    \centering
    \begin{tikzpicture}[scale=0.8]
      \draw[very thick] (0,0.3) -- node [below]{$Q_4$} ++(4,0);
    \end{tikzpicture}
  \end{subfigure}
  \caption*{$
  \color{blue} \overline{Q_5} Q_4 Q_3 Q_2 Q_1 Q_0
  \color{black} +
  \color{red} Q_5 \overline{Q_4}
  \color{black} +
  \color{green} Q_5 \overline{Q_3}
  \color{black} +
  \color{orange} Q_5 \overline{Q_2}
  \color{black} +
  \color{cyan} Q_5 \overline{Q_1}
  \color{black} +
  \color{magenta} Q_5 \overline{Q_0}
  $.}
\end{figure}

\input{asc/q4}
% Copyright 2017 Emilio Rojas
%
% Permission is hereby granted, free of charge, to any person obtaining a copy of
% this software and associated documentation files (the "Software"), to deal in
% the Software without restriction, including without limitation the rights to
% use, copy, modify, merge, publish, distribute, sublicense, and/or sell copies of
% the Software, and to permit persons to whom the Software is furnished to do so,
% subject to the following conditions:
%
% The above copyright notice and this permission notice shall be included in all
% copies or substantial portions of the Software.
%
% THE SOFTWARE IS PROVIDED "AS IS", WITHOUT WARRANTY OF ANY KIND, EXPRESS OR
% IMPLIED, INCLUDING BUT NOT LIMITED TO THE WARRANTIES OF MERCHANTABILITY, FITNESS
% FOR A PARTICULAR PURPOSE AND NONINFRINGEMENT. IN NO EVENT SHALL THE AUTHORS OR
% COPYRIGHT HOLDERS BE LIABLE FOR ANY CLAIM, DAMAGES OR OTHER LIABILITY, WHETHER
% IN AN ACTION OF CONTRACT, TORT OR OTHERWISE, ARISING FROM, OUT OF OR IN
% CONNECTION WITH THE SOFTWARE OR THE USE OR OTHER DEALINGS IN THE SOFTWARE.

\begin{figure}[H]
  \centering

  \begin{subfigure}{0.1\textwidth}
    \centering
    \begin{tikzpicture}[scale=0.8]
      \draw(0,0) -- (0,0);
    \end{tikzpicture}
  \end{subfigure}
  \begin{subfigure}{.4\textwidth}
    \centering
    \begin{Karnaugh}{$Q_3$}{$Q_2$}{$Q_1$}{$Q_0$}
      \minterms{7, 8, 9, 10, 11, 12, 13, 14}
      \implicantsol{7}{blue}
      \implicant{12}{9}{red}
      \implicantcostats[2pt]{12}{10}{green}
      \implicant[5pt]{8}{10}{orange}
    \end{Karnaugh}
  \end{subfigure}
  \begin{subfigure}{.4\textwidth}
    \centering
    \begin{Karnaugh}{$Q_3$}{$Q_2$}{$Q_1$}{$Q_0$}
      \minterms{7, 8, 9, 10, 11, 12, 13, 14}
      \implicantsol{7}{blue}
      \implicant{12}{9}{red}
      \implicantcostats[2pt]{12}{10}{green}
      \implicant[5pt]{8}{10}{orange}
    \end{Karnaugh}
  \end{subfigure}

  \begin{subfigure}{0.1\textwidth}
    \centering
    \begin{tikzpicture}[scale=0.8]
      \draw[very thick] (0,0.3) -- node [left]{$Q_5$} ++(0,4);
    \end{tikzpicture}
  \end{subfigure}
  \begin{subfigure}{.4\textwidth}
    \centering
    \begin{Karnaugh}{$Q_3$}{$Q_2$}{$Q_1$}{$Q_0$}
      \minterms{7, 8, 9, 10, 11, 12, 13, 14}
      \implicantsol{7}{blue}
      \implicant{12}{9}{red}
      \implicantcostats[2pt]{12}{10}{green}
      \implicant[5pt]{8}{10}{orange}
    \end{Karnaugh}
  \end{subfigure}
  \begin{subfigure}{.4\textwidth}
    \centering
    \begin{Karnaugh}{$Q_3$}{$Q_2$}{$Q_1$}{$Q_0$}
      \minterms{7, 8, 9, 10, 11, 12, 13, 14}
      \implicantsol{7}{blue}
      \implicant{12}{9}{red}
      \implicantcostats[2pt]{12}{10}{green}
      \implicant[5pt]{8}{10}{orange}
    \end{Karnaugh}
  \end{subfigure}

  \begin{subfigure}{0.5\textwidth}
    \centering
    \begin{tikzpicture}[scale=0.8]
      \draw(0,0) -- (0,0);
    \end{tikzpicture}
  \end{subfigure}
  \begin{subfigure}{.4\textwidth}
    \centering
    \begin{tikzpicture}[scale=0.8]
      \draw[very thick] (0,0.3) -- node [below]{$Q_4$} ++(4,0);
    \end{tikzpicture}
  \end{subfigure}
  \caption{Mapa de Karnaugh para O del contador ascendente.}
\end{figure}

\input{asc/q2}
\input{asc/q1}
% Copyright 2017 Emilio Rojas
%
% Permission is hereby granted, free of charge, to any person obtaining a copy of
% this software and associated documentation files (the "Software"), to deal in
% the Software without restriction, including without limitation the rights to
% use, copy, modify, merge, publish, distribute, sublicense, and/or sell copies of
% the Software, and to permit persons to whom the Software is furnished to do so,
% subject to the following conditions:
%
% The above copyright notice and this permission notice shall be included in all
% copies or substantial portions of the Software.
%
% THE SOFTWARE IS PROVIDED "AS IS", WITHOUT WARRANTY OF ANY KIND, EXPRESS OR
% IMPLIED, INCLUDING BUT NOT LIMITED TO THE WARRANTIES OF MERCHANTABILITY, FITNESS
% FOR A PARTICULAR PURPOSE AND NONINFRINGEMENT. IN NO EVENT SHALL THE AUTHORS OR
% COPYRIGHT HOLDERS BE LIABLE FOR ANY CLAIM, DAMAGES OR OTHER LIABILITY, WHETHER
% IN AN ACTION OF CONTRACT, TORT OR OTHERWISE, ARISING FROM, OUT OF OR IN
% CONNECTION WITH THE SOFTWARE OR THE USE OR OTHER DEALINGS IN THE SOFTWARE.

\begin{figure}[H]
  \centering
  \caption{Mapa de Karnaugh para $Q_0\prime$ del contador ascendente.}
  \begin{subfigure}{0.1\textwidth}
    \centering
    \begin{tikzpicture}[scale=0.8]
      \draw(0,0) -- (0,0);
    \end{tikzpicture}
  \end{subfigure}
  \begin{subfigure}{.4\textwidth}
    \centering
    \begin{Karnaugh}{$Q_3$}{$Q_2$}{$Q_1$}{$Q_0$}
      \minterms{0,2,4,6,8,10,12,14}
      \implicantcostats{0}{10}{blue}
    \end{Karnaugh}
  \end{subfigure}
  \begin{subfigure}{.4\textwidth}
    \centering
    \begin{Karnaugh}{$Q_3$}{$Q_2$}{$Q_1$}{$Q_0$}
      \minterms{0,2,4,6,8,10,12,14}
      \implicantcostats{0}{10}{blue}
    \end{Karnaugh}
  \end{subfigure}

  \begin{subfigure}{0.1\textwidth}
    \centering
    \begin{tikzpicture}[scale=0.8]
      \draw[very thick] (0,0.3) -- node [left]{$Q_5$} ++(0,4);
    \end{tikzpicture}
  \end{subfigure}
  \begin{subfigure}{.4\textwidth}
    \centering
    \begin{Karnaugh}{$Q_3$}{$Q_2$}{$Q_1$}{$Q_0$}
      \minterms{0,2,4,6,8,10,12,14}
      \implicantcostats{0}{10}{blue}
    \end{Karnaugh}
  \end{subfigure}
  \begin{subfigure}{.4\textwidth}
    \centering
    \begin{Karnaugh}{$Q_3$}{$Q_2$}{$Q_1$}{$Q_0$}
      \minterms{0,2,4,6,8,10,12,14}
      \implicantcostats{0}{10}{blue}
    \end{Karnaugh}
  \end{subfigure}

  \begin{subfigure}{0.5\textwidth}
    \centering
    \begin{tikzpicture}[scale=0.8]
      \draw(0,0) -- (0,0);
    \end{tikzpicture}
  \end{subfigure}
  \begin{subfigure}{.4\textwidth}
    \centering
    \begin{tikzpicture}[scale=0.8]
      \draw[very thick] (0,0.3) -- node [below]{$Q_4$} ++(4,0);
    \end{tikzpicture}
  \end{subfigure}

  \caption*{$\color{blue} \overline{Q_0}$.}
\end{figure}



\usection{Contador Descendente}
Se nota un patrón en las funciones obtenidas para el contador ascendente.
\begin{equation*}
\overline{Q_N} \cdot Q_{N-1} \cdot Q_{N-2} \cdot ... \cdot Q_1 \cdot Q_0 +
Q_N \cdot \overline{Q_{N-1}} +
Q_N \cdot \overline{Q_{N-2}} +
... +
Q_N \cdot \overline{Q_1} +
Q_N \cdot \overline{Q_0}
\end{equation*}
Se asume que sucede lo mismo con
el contador descendente y se resuelven la cantidad de mapas necesarios para
obtener un patrón.

\begin{table}[H]
  \centering
  \begin{adjustbox}{max width=\textwidth}
    \begin{tabular}{ c | c | c | c | c | c | c ||| c | c | c | c | c | c || c | c | c | c | c | c }
       & \multicolumn{6}{c}{\textbf{Número Presente}} &
       \multicolumn{6}{c}{\textbf{Número Siguiente} \small{$Q_5=0$}} &
       \multicolumn{6}{c}{\textbf{Número Siguiente} \small{$Q_5=1$}} \\
      \toprule
      Decimal  &
      $Q_5$ & $Q_4$ & $Q_3$ & $Q_2$ & $Q_1$ & $Q_0$ &
      $Q_5\prime$ & $Q_4\prime$ & $Q_3\prime$ & $Q_2\prime$ & $Q_1\prime$ & $Q_0\prime$ &
      $Q_5\prime$ & $Q_4\prime$ & $Q_3\prime$ & $Q_2\prime$ & $Q_1\prime$ & $Q_0\prime$ \\
      \toprule
      %       Q5    Q4  Q3  Q2  Q1  Q0    Q5  Q4  Q3  Q2  Q1  Q0    Q5  Q4  Q3  Q2  Q1  Q0
      0 /32 & 0/1 & 0 & 0 & 0 & 0 & 0 &   0 & 0 & 0 & 0 & 0 & 1 &   1 & 0 & 0 & 0 & 0 & 1 \\
      %       Q5    Q4  Q3  Q2  Q1  Q0    Q5  Q4  Q3  Q2  Q1  Q0    Q5  Q4  Q3  Q2  Q1  Q0
      1 /33 & 0/1 & 0 & 0 & 0 & 0 & 1 &   0 & 0 & 0 & 0 & 1 & 0 &   1 & 0 & 0 & 0 & 1 & 0 \\
      %       Q5    04  Q3  Q2  Q1  Q0    Q5  Q4  Q3  Q2  Q1  Q0    Q5  Q4  Q3  Q2  Q1  Q0
      2 /34 & 0/1 & 0 & 0 & 0 & 1 & 0 &   0 & 0 & 0 & 0 & 1 & 1 &   1 & 0 & 0 & 0 & 1 & 1 \\
      %       Q5    Q4  Q3  Q2  Q1  Q0    Q5  Q4  Q3  Q2  Q1  Q0    Q5  Q4  Q3  Q2  Q1  Q0
      3 /35 & 0/1 & 0 & 0 & 0 & 1 & 1 &   0 & 0 & 0 & 1 & 0 & 0 &   1 & 0 & 0 & 1 & 0 & 0 \\ \hline
      %       Q5    Q4  Q3  Q2  Q1  Q0    Q5  Q4  Q3  Q2  Q1  Q0    Q5  Q4  Q3  Q2  Q1  Q0
      4 /36 & 0/1 & 0 & 0 & 1 & 0 & 0 &   0 & 0 & 0 & 1 & 0 & 1 &   1 & 0 & 0 & 1 & 0 & 1 \\
      %       Q5    Q4  Q3  Q2  Q1  Q0    Q5  Q4  Q3  Q2  Q1  Q0    Q5  Q4  Q3  Q2  Q1  Q0
      5 /37 & 0/1 & 0 & 0 & 1 & 0 & 1 &   0 & 0 & 0 & 1 & 1 & 0 &   1 & 0 & 0 & 1 & 1 & 0 \\
      %       Q5    Q4  Q3  Q2  Q1  Q0    Q5  Q4  Q3  Q2  Q1  Q0    Q5  Q4  Q3  Q2  Q1  Q0
      6 /38 & 0/1 & 0 & 0 & 1 & 1 & 0 &   0 & 0 & 0 & 1 & 1 & 1 &   1 & 0 & 0 & 1 & 1 & 1 \\
      %       Q5    Q4  Q3  Q2  Q1  Q0    Q5  Q4  Q3  Q2  Q1  Q0    Q5  Q4  Q3  Q2  Q1  Q0
      7 /39 & 0/1 & 0 & 0 & 1 & 1 & 1 &   0 & 0 & 1 & 0 & 0 & 0 &   1 & 0 & 1 & 0 & 0 & 0 \\ \hline
      %       Q5    Q4  Q3  Q2  Q1  Q0    Q5  Q4  Q3  Q2  Q1  Q0    Q5  Q4  Q3  Q2  Q1  Q0
      8 /40 & 0/1 & 0 & 1 & 0 & 0 & 0 &   0 & 0 & 1 & 0 & 0 & 1 &   1 & 0 & 1 & 0 & 0 & 1 \\
      %       Q5    Q4  Q3  Q2  Q1  Q0    Q5  Q4  Q3  Q2  Q1  Q0    Q5  Q4  Q3  Q2  Q1  Q0
      9 /41 & 0/1 & 0 & 1 & 0 & 0 & 1 &   0 & 0 & 1 & 0 & 1 & 0 &   1 & 0 & 1 & 0 & 1 & 0 \\
      %       Q5    Q4  Q3  Q2  Q1  Q0    Q5  Q4  Q3  Q2  Q1  Q0    Q5  Q4  Q3  Q2  Q1  Q0
      10/42 & 0/1 & 0 & 1 & 0 & 1 & 0 &   0 & 0 & 1 & 0 & 1 & 1 &   1 & 0 & 1 & 0 & 1 & 1 \\
      %       Q5    Q4  Q3  Q2  Q1  Q0    Q5  Q4  Q3  Q2  Q1  Q0    Q5  Q4  Q3  Q2  Q1  Q0
      11/43 & 0/1 & 0 & 1 & 0 & 1 & 1 &   0 & 0 & 1 & 1 & 0 & 0 &   1 & 0 & 1 & 1 & 0 & 0 \\ \hline
      %       Q5    Q4  Q3  Q2  Q1  Q0    Q5  Q4  Q3  Q2  Q1  Q0    Q5  Q4  Q3  Q2  Q1  Q0
      12/44 & 0/1 & 0 & 1 & 1 & 0 & 0 &   0 & 0 & 1 & 1 & 0 & 1 &   1 & 0 & 1 & 1 & 0 & 1 \\
      %       Q5    Q4  Q3  Q2  Q1  Q0    Q5  Q4  Q3  Q2  Q1  Q0    Q5  Q4  Q3  Q2  Q1  Q0
      13/45 & 0/1 & 0 & 1 & 1 & 0 & 1 &   0 & 0 & 1 & 1 & 1 & 0 &   1 & 0 & 1 & 1 & 1 & 0 \\
      %       Q5    Q4  Q3  Q2  Q1  Q0    Q5  Q4  Q3  Q2  Q1  Q0    Q5  Q4  Q3  Q2  Q1  Q0
      14/46 & 0/1 & 0 & 1 & 1 & 1 & 0 &   0 & 0 & 1 & 1 & 1 & 1 &   1 & 0 & 1 & 1 & 1 & 1 \\
      %       Q5    Q4  Q3  Q2  Q1  Q0    Q5  Q4  Q3  Q2  Q1  Q0    Q5  Q4  Q3  Q2  Q1  Q0
      15/47 & 0/1 & 0 & 1 & 1 & 1 & 1 &   0 & 1 & 0 & 0 & 0 & 0 &   1 & 1 & 0 & 0 & 0 & 0 \\ \hline
      %       Q5    Q4  Q3  Q2  Q1  Q0    Q5  Q4  Q3  Q2  Q1  Q0    Q5  Q4  Q3  Q2  Q1  Q0
      16/48 & 0/1 & 1 & 0 & 0 & 0 & 0 &   0 & 1 & 0 & 0 & 0 & 1 &   1 & 1 & 0 & 0 & 0 & 1 \\
      %       Q5    Q4  Q3  Q2  Q1  Q0    Q5  Q4  Q3  Q2  Q1  Q0    Q5  Q4  Q3  Q2  Q1  Q0
      17/49 & 0/1 & 1 & 0 & 0 & 0 & 1 &   0 & 1 & 0 & 0 & 1 & 0 &   1 & 1 & 0 & 0 & 1 & 0 \\
      %       Q5    Q4  Q3  Q2  Q1  Q0    Q5  Q4  Q3  Q2  Q1  Q0    Q5  Q4  Q3  Q2  Q1  Q0
      18/50 & 0/1 & 1 & 0 & 0 & 1 & 0 &   0 & 1 & 0 & 0 & 1 & 1 &   1 & 1 & 0 & 0 & 1 & 1 \\
      %       Q5    Q4  Q3  Q2  Q1  Q0    Q5  Q4  Q3  Q2  Q1  Q0    Q5  Q4  Q3  Q2  Q1  Q0
      19/51 & 0/1 & 1 & 0 & 0 & 1 & 1 &   0 & 1 & 0 & 1 & 0 & 0 &   1 & 1 & 0 & 1 & 0 & 0 \\ \hline
      %       Q5    Q4  Q3  Q2  Q1  Q0    Q5  Q4  Q3  Q2  Q1  Q0    Q5  Q4  Q3  Q2  Q1  Q0
      20/52 & 0/1 & 1 & 0 & 1 & 0 & 0 &   0 & 1 & 0 & 1 & 0 & 1 &   1 & 1 & 0 & 1 & 0 & 1 \\
      %       Q5    Q4  Q3  Q2  Q1  Q0    Q5  Q4  Q3  Q2  Q1  Q0    Q5  Q4  Q3  Q2  Q1  Q0
      21/53 & 0/1 & 1 & 0 & 1 & 0 & 1 &   0 & 1 & 0 & 1 & 1 & 0 &   1 & 1 & 0 & 1 & 1 & 0 \\
      %       Q5    Q4  Q3  Q2  Q1  Q0    Q5  Q4  Q3  Q2  Q1  Q0    Q5  Q4  Q3  Q2  Q1  Q0
      22/54 & 0/1 & 1 & 0 & 1 & 1 & 0 &   0 & 1 & 0 & 1 & 1 & 1 &   1 & 1 & 0 & 1 & 1 & 1 \\
      %       Q5    Q4  Q3  Q2  Q1  Q0    Q5  Q4  Q3  Q2  Q1  Q0    Q5  Q4  Q3  Q2  Q1  Q0
      23/55 & 0/1 & 1 & 0 & 1 & 1 & 1 &   0 & 1 & 1 & 0 & 0 & 0 &   1 & 1 & 1 & 0 & 0 & 0 \\ \hline
      %       Q5    Q4  Q3  Q2  Q1  Q0    Q5  Q4  Q3  Q2  Q1  Q0    Q5  Q4  Q3  Q2  Q1  Q0
      24/56 & 0/1 & 1 & 1 & 0 & 0 & 0 &   0 & 1 & 1 & 0 & 0 & 1 &   1 & 1 & 1 & 0 & 0 & 1 \\
      %       Q5    Q4  Q3  Q2  Q1  Q0    Q5  Q4  Q3  Q2  Q1  Q0    Q5  Q4  Q3  Q2  Q1  Q0
      25/57 & 0/1 & 1 & 1 & 0 & 0 & 1 &   0 & 1 & 1 & 0 & 1 & 0 &   1 & 1 & 1 & 0 & 1 & 0 \\
      %       Q5    Q4  Q3  Q2  Q1  Q0    Q5  Q4  Q3  Q2  Q1  Q0    Q5  Q4  Q3  Q2  Q1  Q0
      26/58 & 0/1 & 1 & 1 & 0 & 1 & 0 &   0 & 1 & 1 & 0 & 1 & 1 &   1 & 1 & 1 & 0 & 1 & 1 \\
      %       Q5    Q4  Q3  Q2  Q1  Q0    Q5  Q4  Q3  Q2  Q1  Q0    Q5  Q4  Q3  Q2  Q1  Q0
      27/59 & 0/1 & 1 & 1 & 0 & 1 & 1 &   0 & 1 & 1 & 1 & 0 & 0 &   1 & 1 & 1 & 1 & 0 & 0 \\ \hline
      %       Q5    Q4  Q3  Q2  Q1  Q0    Q5  Q4  Q3  Q2  Q1  Q0    Q5  Q4  Q3  Q2  Q1  Q0
      28/60 & 0/1 & 1 & 1 & 1 & 0 & 0 &   0 & 1 & 1 & 1 & 0 & 1 &   1 & 1 & 1 & 1 & 0 & 1 \\
      %       Q5    Q4  Q3  Q2  Q1  Q0    Q5  Q4  Q3  Q2  Q1  Q0    Q5  Q4  Q3  Q2  Q1  Q0
      29/61 & 0/1 & 1 & 1 & 1 & 0 & 1 &   0 & 1 & 1 & 1 & 1 & 0 &   1 & 1 & 1 & 1 & 1 & 0 \\
      %       Q5    Q4  Q3  Q2  Q1  Q0    Q5  Q4  Q3  Q2  Q1  Q0    Q5  Q4  Q3  Q2  Q1  Q0
      30/62 & 0/1 & 1 & 1 & 1 & 1 & 0 &   0 & 1 & 1 & 1 & 1 & 1 &   1 & 1 & 1 & 1 & 1 & 1 \\
      %       Q5    Q4  Q3  Q2  Q1  Q0    Q5  Q4  Q3  Q2  Q1  Q0    Q5  Q4  Q3  Q2  Q1  Q0
      31/63 & 0/1 & 1 & 1 & 1 & 1 & 1 &   1 & 0 & 0 & 0 & 0 & 0 &   0 & 0 & 0 & 0 & 0 & 0 \\ \bottomrule
      %       Q5    Q4  Q3  Q2  Q1  Q0    Q5  Q4  Q3  Q2  Q1  Q0    Q5  Q4  Q3  Q2  Q1  Q0
    \end{tabular}
  \end{adjustbox}
  \caption{Tabla de verdad para el número siguiente de 6bits. Se muestra del 0 al
  31 en la columna \textbf{Número Siguiente} \small{$Q_5=0$}, del 32 al 63 en la
  columna \textbf{Número Siguiente} \small{$Q_5=1$}}
\end{table}

% Copyright 2017 Emilio Rojas
%
% Permission is hereby granted, free of charge, to any person obtaining a copy of
% this software and associated documentation files (the "Software"), to deal in
% the Software without restriction, including without limitation the rights to
% use, copy, modify, merge, publish, distribute, sublicense, and/or sell copies of
% the Software, and to permit persons to whom the Software is furnished to do so,
% subject to the following conditions:
%
% The above copyright notice and this permission notice shall be included in all
% copies or substantial portions of the Software.
%
% THE SOFTWARE IS PROVIDED "AS IS", WITHOUT WARRANTY OF ANY KIND, EXPRESS OR
% IMPLIED, INCLUDING BUT NOT LIMITED TO THE WARRANTIES OF MERCHANTABILITY, FITNESS
% FOR A PARTICULAR PURPOSE AND NONINFRINGEMENT. IN NO EVENT SHALL THE AUTHORS OR
% COPYRIGHT HOLDERS BE LIABLE FOR ANY CLAIM, DAMAGES OR OTHER LIABILITY, WHETHER
% IN AN ACTION OF CONTRACT, TORT OR OTHERWISE, ARISING FROM, OUT OF OR IN
% CONNECTION WITH THE SOFTWARE OR THE USE OR OTHER DEALINGS IN THE SOFTWARE.

\begin{figure}[H]
  \centering
  \caption{Mapa de Karnaugh para $Q_0\prime$ del contador ascendente.}
  \begin{subfigure}{0.1\textwidth}
    \centering
    \begin{tikzpicture}[scale=0.8]
      \draw(0,0) -- (0,0);
    \end{tikzpicture}
  \end{subfigure}
  \begin{subfigure}{.4\textwidth}
    \centering
    \begin{Karnaugh}{$Q_3$}{$Q_2$}{$Q_1$}{$Q_0$}
      \minterms{0,2,4,6,8,10,12,14}
      \implicantcostats{0}{10}{blue}
    \end{Karnaugh}
  \end{subfigure}
  \begin{subfigure}{.4\textwidth}
    \centering
    \begin{Karnaugh}{$Q_3$}{$Q_2$}{$Q_1$}{$Q_0$}
      \minterms{0,2,4,6,8,10,12,14}
      \implicantcostats{0}{10}{blue}
    \end{Karnaugh}
  \end{subfigure}

  \begin{subfigure}{0.1\textwidth}
    \centering
    \begin{tikzpicture}[scale=0.8]
      \draw[very thick] (0,0.3) -- node [left]{$Q_5$} ++(0,4);
    \end{tikzpicture}
  \end{subfigure}
  \begin{subfigure}{.4\textwidth}
    \centering
    \begin{Karnaugh}{$Q_3$}{$Q_2$}{$Q_1$}{$Q_0$}
      \minterms{0,2,4,6,8,10,12,14}
      \implicantcostats{0}{10}{blue}
    \end{Karnaugh}
  \end{subfigure}
  \begin{subfigure}{.4\textwidth}
    \centering
    \begin{Karnaugh}{$Q_3$}{$Q_2$}{$Q_1$}{$Q_0$}
      \minterms{0,2,4,6,8,10,12,14}
      \implicantcostats{0}{10}{blue}
    \end{Karnaugh}
  \end{subfigure}

  \begin{subfigure}{0.5\textwidth}
    \centering
    \begin{tikzpicture}[scale=0.8]
      \draw(0,0) -- (0,0);
    \end{tikzpicture}
  \end{subfigure}
  \begin{subfigure}{.4\textwidth}
    \centering
    \begin{tikzpicture}[scale=0.8]
      \draw[very thick] (0,0.3) -- node [below]{$Q_4$} ++(4,0);
    \end{tikzpicture}
  \end{subfigure}

  \caption*{$\color{blue} \overline{Q_0}$.}
\end{figure}

\input{desc/q1}
\input{desc/q2}

Se nota un patrón en las funciones obtenidas para el contador descendente hasta el momento.
\begin{equation*}
\overline{Q_N} \cdot \overline{Q_{N-1}} \cdot \overline{Q_{N-2}} \cdot ... \cdot \overline{Q_1} \cdot \overline{Q_0} +
Q_N \cdot Q_{N-1} +
Q_N \cdot Q_{N-2} +
... +
Q_N \cdot Q_1 +
Q_N \cdot Q_0
\end{equation*}

De esto se obtiene para O:
\begin{equation*}
\overline{Q_3} \phantom{\cdot} \overline{Q_2} \phantom{\cdot} \overline{Q_1} \phantom{\cdot} \overline{Q_0} +
Q_3 \phantom{\cdot} Q_2 +
Q_3 \phantom{\cdot} Q_1 +
Q_3 \phantom{\cdot} Q_0
\end{equation*}

Para N:
\begin{equation*}
\overline{Q_4} \phantom{\cdot} \overline{Q_3} \phantom{\cdot} \overline{Q_2} \phantom{\cdot} \overline{Q_1} \phantom{\cdot} \overline{Q_0} +
Q_4 \phantom{\cdot} Q_3 +
Q_4 \phantom{\cdot} Q_2 +
Q_4 \phantom{\cdot} Q_1 +
Q_4 \phantom{\cdot} Q_0
\end{equation*}

Para M:
\begin{equation*}
\overline{Q_5} \phantom{\cdot} \overline{Q_4} \phantom{\cdot} \overline{Q_3} \phantom{\cdot} \overline{Q_2} \phantom{\cdot} \overline{Q_1} \phantom{\cdot} \overline{Q_0} +
Q_5 \phantom{\cdot} Q_4 +
Q_5 \phantom{\cdot} Q_3 +
Q_5 \phantom{\cdot} Q_2 +
Q_5 \phantom{\cdot} Q_1 +
Q_5 \phantom{\cdot} Q_0
\end{equation*}

\end{document}
