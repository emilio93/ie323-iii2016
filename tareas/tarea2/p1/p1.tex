  \usubsubsection{Pantallas de 7 Segmentos}

    \begin{figure}[H]
      \begin{center}
        \begin{tikzpicture}
          \SSGLeg[3em]{}
        \end{tikzpicture}
      \end{center}

      \begin{center}
        \begin{tikzpicture}
          \coordinate(A1)at(0,0);
          \coordinate(A2)at(2,0);
          \coordinate(A3)at(4,0);
          \coordinate(A4)at(6,0);
          \coordinate(A5)at(8,0);
          \SSGNb[2em]{A1}{0}
          \SSGNb[2em]{A2}{1}
          \SSGNb[2em]{A3}{2}
          \SSGNb[2em]{A4}{3}
          \SSGNb[2em]{A5}{4}
        \end{tikzpicture}
      \end{center}

      \begin{center}
        \begin{tikzpicture}
          \coordinate(A1)at(0,0);
          \coordinate(A2)at(2,0);
          \coordinate(A3)at(4,0);
          \coordinate(A4)at(6,0);
          \coordinate(A5)at(8,0);
          \SSGNb[2em]{A1}{5}
          \SSGNb[2em]{A2}{6}
          \SSGNb[2em]{A3}{7}
          \SSGNb[2em]{A4}{8}
          \SSGNb[2em]{A5}{9}
        \end{tikzpicture}
      \end{center}

      \begin{center}
        \begin{tikzpicture}
          \coordinate(A1)at(0,0);
          \coordinate(A2)at(2,0);
          \coordinate(A3)at(4,0);
          \coordinate(A4)at(6,0);
          \SSGNb[2em]{A1}{H}
          \SSGNb[2em]{A2}{O}
          \SSGNb[2em]{A3}{L}
          \SSGNb[2em]{A4}{A}
        \end{tikzpicture}
      \end{center}
      \label{imseg}
      \caption{Estados de la pantalla de 7 segmentos}
    \end{figure}

  \usubsubsection{Tabla de Verdad}
    \begin{table}[H]
      \begin{center}
        \begin{tabular}{ c ||| c  c  c  c ||| c | c | c | c | c | c | c}
          \toprule
          Valor & X & Y & Z & W & a & b & c & d & e & f & g \\
          \toprule
          %         X & Y & Z & W         a   b   c   d   e   f   g
          0    &    0 & 0 & 0 & 0    &    1 & 1 & 1 & 1 & 1 & 1 & 0 \\
          1    &    0 & 0 & 0 & 1    &    0 & 1 & 1 & 0 & 0 & 0 & 0 \\
          % \midrule
          %         X & Y & Z & W         a   b   c   d   e   f   g
          2    &    0 & 0 & 1 & 0    &    1 & 1 & 0 & 1 & 1 & 0 & 1 \\
          3    &    0 & 0 & 1 & 1    &    1 & 1 & 1 & 1 & 0 & 0 & 1 \\
          \midrule
          %         X & Y & Z & W         a   b   c   d   e   f   g
          4    &    0 & 1 & 0 & 0    &    0 & 1 & 1 & 0 & 0 & 1 & 1 \\
          5    &    0 & 1 & 0 & 1    &    1 & 0 & 1 & 1 & 0 & 1 & 1 \\
          % \midrule
          %         X & Y & Z & W         a   b   c   d   e   f   g
          6    &    0 & 1 & 1 & 0    &    1 & 0 & 1 & 1 & 1 & 1 & 1 \\
          7    &    0 & 1 & 1 & 1    &    1 & 1 & 1 & 0 & 0 & 0 & 0 \\
          \midrule
          %         X & Y & Z & W         a   b   c   d   e   f   g
          8    &    1 & 0 & 0 & 0    &    1 & 1 & 1 & 1 & 1 & 1 & 1 \\
          9    &    1 & 0 & 0 & 1    &    1 & 1 & 1 & 1 & 0 & 1 & 1 \\
          % \midrule
          %         X & Y & Z & W         a   b   c   d   e   f   g
          H    &    1 & 0 & 1 & 0    &    0 & 1 & 1 & 0 & 1 & 1 & 1 \\
          O    &    1 & 0 & 1 & 1    &    1 & 1 & 1 & 1 & 1 & 1 & 0 \\
          \midrule
          %         X & Y & Z & W         a   b   c   d   e   f   g
          L    &    1 & 1 & 0 & 0    &    0 & 0 & 0 & 1 & 1 & 1 & 0 \\
          A    &    1 & 1 & 0 & 1    &    1 & 1 & 1 & 0 & 1 & 1 & 1 \\
          % \midrule
          %         X & Y & Z & W         a   b   c   d   e   f   g
          X    &    1 & 1 & 1 & 0    &    X & X & X & X & X & X & X \\
          X    &    1 & 1 & 1 & 1    &    X & X & X & X & X & X & X \\
          \bottomrule
        \end{tabular}
        \label{tab:verdad}
        \caption{Tabla de Verdad}
      \end{center}
    \end{table}

  \usubsubsection{Mapas de Karnaugh}
    Los mapas de Karnaugh se muestran según el siguiente ejemplo que muestra el
    número correspondiente de cada celda.
    \begin{center}
      \begin{Karnaugh}{X}{Y}{Z}{W}
        \contingut{
        0, 1, 2, 3,
        4, 5, 6, 7,
        8, 9, 10,11,
        12,13,14,15
        }
      \end{Karnaugh}
    \end{center}

    Para el segmento a:
%\begin{equation*}
%  F = \sum \text{m}(0, 2, 3, 5, 6, 7, 8, 9, 11, 13) + \text{d}(14, 15)
%\end{equation*}
\begin{figure}[H]
  \centering
    \begin{Karnaugh}{X}{Y}{Z}{W}
      \contingut{
        1,
        ,
        1,
        1,
        ,
        1,
        1,
        1,
        1,
        1,
        ,
        1,
        ,
        1,
        X,
        X
      }
      \implicantdaltbaix{0}{8}{black}
      \implicantcostats[2pt]{0}{2}{orange}
      \implicant{3}{6}{green}
      \implicant[5pt]{3}{11}{cyan}
      \implicant[2pt]{5}{15}{blue}
      \implicant[2pt]{8}{9}{magenta}
      \implicant{13}{11}{red}
    \end{Karnaugh}
\end{figure}

\begin{align*}
  \bar{Y} \cdot \bar{Z} \cdot \bar{W} & &
  \color{orange} \bar{X} \cdot \bar{Y} \cdot \bar{W}& &
  \color{green} \bar{X} \cdot Z & (*) &
  \color{cyan} Z \cdot W \\
  \color{blue} Y \cdot W & (*) &
  \color{magenta} X \cdot \bar{Y} \cdot \bar{Z} & &
  \color{red} X \cdot W &
  \color{black}
\end{align*}

\begin{equation*}
  \bar{Y} \cdot \bar{Z} \cdot \bar{W} +
  \bar{X} \cdot Z +
  Y \cdot W +
  X \cdot W
\end{equation*}

    Para el segmento b:
\begin{figure}[H]
  \centering
    \begin{Karnaugh}{X}{Y}{Z}{W}
      \contingut{
        1,
        1,
        1,
        1,
        1,
        ,
        ,
        1,
        1,
        1,
        1,
        1,
        ,
        1,
        X,
        X
      }
      \implicantdaltbaix[4pt]{0}{10}{black}
      \implicant{0}{4}{green}
      \implicant{3}{11}{blue}
      \implicant[4pt]{13}{11}{red}
    \end{Karnaugh}
\end{figure}

\begin{align*}
  \bar{Y} & (*) &
  \color{green} \bar{X} \cdot \bar{Z} \cdot \bar{W}& (*) &
  \color{blue} Z \cdot W & (*) &
  \color{red} X \cdot W & (*)
  \color{black}
\end{align*}

\begin{equation*}
  \bar{Y} +
  \bar{X} \cdot \bar{Z} \cdot \bar{W}+
  Z \cdot W +
  X \cdot W
\end{equation*}

    Para el segmento c:
\begin{figure}[H]
  \centering
    \begin{Karnaugh}{X}{Y}{Z}{W}
      \contingut{
        1,
        1,
        ,
        1,
        1,
        1,
        1,
        1,
        1,
        1,
        1,
        1,
        ,
        1,
        X,
        X
      }
      \implicant{0}{5}{black}
      \implicant[4pt]{1}{11}{green}
      \implicant[2pt]{7}{14}{blue}
      \implicant{8}{10}{red}
      \implicant[4pt]{4}{6}{orange}
    \end{Karnaugh}
\end{figure}

\begin{align*}
  \bar{X} \cdot \bar{W} & (*) &
  \color{green} W & (*) &
  \color{blue} Y \cdot Z & &
  \color{red} X \cdot \bar{Y} & (*) \\
  \color{orange} \bar{X}\cdot Y
  \color{black}
\end{align*}

\begin{equation*}
  \bar{X} \cdot \bar{W} +
  W +
  X \cdot \bar{Y} +
  \bar{X}\cdot Y
\end{equation*}

    Para el segmento d:
\begin{figure}[H]
  \centering
    \begin{Karnaugh}{X}{Y}{Z}{W}
      \contingut{
        1,
        ,
        1,
        1,
        ,
        1,
        1,
        ,
        1,
        1,
        ,
        1,
        1,
        ,
        X,
        X
      }
      \implicantdaltbaix[4pt]{0}{8}{black}
      \implicant[2pt]{3}{2}{yellow}
      \implicantdaltbaix[4pt]{3}{11}{green}
      \implicantcostats{0}{2}{blue}
      \implicant[4pt]{2}{6}{red}
      \implicant{5}{5}{orange}
      \implicant{8}{9}{cyan}
      \implicant[2pt]{9}{11}{magenta}
      \implicantcostats{12}{14}{purple}
      \implicant[2pt]{12}{8}{pink}
    \end{Karnaugh}
\end{figure}

\begin{align*}
  \bar{Y} \cdot \bar{Z} \cdot \bar{W} & &
  \color{green} \bar{Y} \cdot Z \cdot W & &
  \color{blue} \bar{X} \cdot \bar{Y} \cdot \bar{W} & &
  \color{red} \bar{X} \cdot Z \cdot \bar{W} & (*) \\
  \color{orange} \bar{X} \cdot Y \cdot \bar{Z} \cdot W & (*) &
  \color{cyan} X \cdot \bar{Y} \cdot \bar{Z} & &
  \color{magenta} X \cdot \bar{Y} \cdot W & &
  \color{purple} X \cdot Y \cdot \bar{W} \\
  \color{pink} X \cdot \bar{Z} \cdot \bar{W} & &
\end{align*}

\begin{equation*}
  \bar{Y} \cdot \bar{Z} \cdot \bar{W} +
  \bar{Y} \cdot Z \cdot W +
  \bar{X} \cdot Z \cdot \bar{W} +
  X \cdot \bar{Z} \cdot \bar{W} +
  X \cdot \bar{Y} \cdot W +
  \bar{X} \cdot Y \cdot \bar{Z} \cdot W
\end{equation*}

    Para el segmento e:
\begin{figure}[H]
  \begin{center}
    \begin{Karnaugh}{X}{Y}{Z}{W}
      \contingut{
        1,
        ,
        1,
        ,
        ,
        ,
        1,
        ,
        1,
        ,
        1,
        1,
        1,
        1,
        X,
        X
      }
      \implicantcantons{black}
      \implicant[2pt]{2}{10}{blue}
      \implicant[4pt]{12}{14}{green}
      \implicant{15}{10}{red}
      \implicant{12}{8}{orange}
    \end{Karnaugh}
  \end{center}
\end{figure}

    Para el segmento f:
\begin{figure}[H]
  \begin{center}
    \begin{Karnaugh}{X}{Y}{Z}{W}
      \contingut{
      1,
      ,
      ,
      ,
      1,
      1,
      1,
      ,
      1,
      1,
      1,
      1,
      1,
      1,
      X,
      X
      }
      \implicant{0}{8}{black}
      \implicant[2pt]{12}{10}{blue}
      \implicant[4pt]{4}{13}{green}
      \implicantcostats[2pt]{4}{6}{red}
    \end{Karnaugh}
  \end{center}
\end{figure}

\begin{align*}
  \bar{Z} \cdot \bar{W} & (*) &
  \color{green} Y \cdot \bar{Z} & (*) &
  \color{blue} X & (*) &
  \color{red} \bar{X} \cdot Y \cdot \bar{W} & (*)
\end{align*}

\begin{equation*}
  \bar{Z} \cdot \bar{W} +
  Y \cdot \bar{Z} +
  X +
  \bar{X} \cdot Y \cdot \bar{W}
\end{equation*}

    Para el segmento g:
\begin{figure}[H]
  \begin{center}
    \begin{Karnaugh}{X}{Y}{Z}{W}
      \contingut{
        ,
        ,
        1,
        1,
        1,
        1,
        1,
        ,
        1,
        1,
        1,
        ,
        ,
        1,
        X,
        X
      }
    \end{Karnaugh}
  \end{center}
\end{figure}

