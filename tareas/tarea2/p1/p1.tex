\usection{Problema 1}

  \usubsection{Enunciado}

  \usubsection{Solución}
  \usubsubsection{Pantallas de 7 Segmentos}

    \begin{figure}[H]
      \begin{center}
        \begin{tikzpicture}
          \SSGLeg[3em]{}
        \end{tikzpicture}
      \end{center}

      \begin{center}
        \begin{tikzpicture}
          \coordinate(A1)at(0,0);
          \coordinate(A2)at(2,0);
          \coordinate(A3)at(4,0);
          \coordinate(A4)at(6,0);
          \coordinate(A5)at(8,0);
          \SSGNb[2em]{A1}{0}
          \SSGNb[2em]{A2}{1}
          \SSGNb[2em]{A3}{2}
          \SSGNb[2em]{A4}{3}
          \SSGNb[2em]{A5}{4}
        \end{tikzpicture}
      \end{center}

      \begin{center}
        \begin{tikzpicture}
          \coordinate(A1)at(0,0);
          \coordinate(A2)at(2,0);
          \coordinate(A3)at(4,0);
          \coordinate(A4)at(6,0);
          \coordinate(A5)at(8,0);
          \SSGNb[2em]{A1}{5}
          \SSGNb[2em]{A2}{6}
          \SSGNb[2em]{A3}{7}
          \SSGNb[2em]{A4}{8}
          \SSGNb[2em]{A5}{9}
        \end{tikzpicture}
      \end{center}

      \begin{center}
        \begin{tikzpicture}
          \coordinate(A1)at(0,0);
          \coordinate(A2)at(2,0);
          \coordinate(A3)at(4,0);
          \coordinate(A4)at(6,0);
          \SSGNb[2em]{A1}{H}
          \SSGNb[2em]{A2}{O}
          \SSGNb[2em]{A3}{L}
          \SSGNb[2em]{A4}{A}
        \end{tikzpicture}
      \end{center}
      \label{imseg}
      \caption{Estados de la pantalla de 7 segmentos}
    \end{figure}
    Se utiliza la configuración mostrada en la primer imagen de la Figura \ref{imseg}

  \usubsubsection{Tabla de Verdad}
  \begin{table}[H]
    \begin{center}
      \begin{tabular}{ c ||| c  c  c  c ||| c | c | c | c | c | c | c}
        \toprule
        Valor & X & Y & Z & W & a & b & c & d & e & f & g \\
        \toprule
        %         X & Y & Z & W         a   b   c   d   e   f   g
        0    &    0 & 0 & 0 & 0    &    1 & 1 & 1 & 1 & 1 & 1 & 0 \\
        1    &    0 & 0 & 0 & 1    &    0 & 1 & 1 & 0 & 0 & 0 & 0 \\
        % \midrule
        %         X & Y & Z & W         a   b   c   d   e   f   g
        2    &    0 & 0 & 1 & 0    &    1 & 1 & 0 & 1 & 1 & 0 & 1 \\
        3    &    0 & 0 & 1 & 1    &    1 & 1 & 1 & 1 & 0 & 0 & 1 \\
        \midrule
        %         X & Y & Z & W         a   b   c   d   e   f   g
        4    &    0 & 1 & 0 & 0    &    0 & 1 & 1 & 0 & 0 & 1 & 1 \\
        5    &    0 & 1 & 0 & 1    &    1 & 0 & 1 & 1 & 0 & 1 & 1 \\
        % \midrule
        %         X & Y & Z & W         a   b   c   d   e   f   g
        6    &    0 & 1 & 1 & 0    &    1 & 0 & 1 & 1 & 1 & 1 & 1 \\
        7    &    0 & 1 & 1 & 1    &    1 & 1 & 1 & 0 & 0 & 0 & 0 \\
        \midrule
        %         X & Y & Z & W         a   b   c   d   e   f   g
        8    &    1 & 0 & 0 & 0    &    1 & 1 & 1 & 1 & 1 & 1 & 1 \\
        9    &    1 & 0 & 0 & 1    &    1 & 1 & 1 & 1 & 0 & 1 & 1 \\
        % \midrule
        %         X & Y & Z & W         a   b   c   d   e   f   g
        H    &    1 & 0 & 1 & 0    &    0 & 1 & 1 & 0 & 1 & 1 & 1 \\
        O    &    1 & 0 & 1 & 1    &    1 & 1 & 1 & 1 & 1 & 1 & 0 \\
        \midrule
        %         X & Y & Z & W         a   b   c   d   e   f   g
        L    &    1 & 1 & 0 & 0    &    0 & 0 & 0 & 1 & 1 & 1 & 0 \\
        A    &    1 & 1 & 0 & 1    &    1 & 1 & 1 & 0 & 1 & 1 & 1 \\
        % \midrule
        %         X & Y & Z & W         a   b   c   d   e   f   g
        X    &    1 & 1 & 1 & 0    &    X & X & X & X & X & X & X \\
        X    &    1 & 1 & 1 & 1    &    X & X & X & X & X & X & X \\
        \bottomrule
      \end{tabular}
      \label{tab:verdad}
      \caption{Tabla de Verdad}
    \end{center}
  \end{table}


  \usubsubsection{Mapas de Karnaugh}

  Los mapas de Karnaugh se muestran según el siguiente ejemplo que muestra el
  número correspondiente de cada celda.
  \begin{center}
    \begin{Karnaugh}{X}{Y}{Z}{W}
      \contingut{
      0, 1, 2, 3,
      4, 5, 6, 7,
      8, 9, 10,11,
      12,13,14,15
      }
    \end{Karnaugh}
  \end{center}

  Para el segmento a:
  \begin{figure}[H]
    \begin{center}
      \begin{Karnaugh}{X}{Y}{Z}{W}
        \contingut{
          1,
          ,
          1,
          1,
          ,
          1,
          1,
          1,
          1,
          1,
          ,
          1,
          ,
          1,
          X,
          X
        }
        \implicantdaltbaix{0}{8}{black}
        \implicantcostats[2pt]{0}{2}{orange}
        \implicant{3}{6}{green}
        \implicant[2pt]{8}{9}{yellow}
        \implicant[5pt]{3}{11}{pink}
        \implicant[2pt]{5}{15}{blue}
        \implicant{13}{11}{red}
      \end{Karnaugh}
    \end{center}
  \end{figure}

  Para el segmento b:
  \begin{figure}[H]
    \begin{center}
      \begin{Karnaugh}{X}{Y}{Z}{W}
        \contingut{
          1,
          1,
          1,
          1,
          1,
          ,
          ,
          1,
          1,
          1,
          1,
          1,
          ,
          1,
          X,
          X
        }
        \implicantdaltbaix[4pt]{0}{10}{black}
        \implicant{0}{4}{green}
        \implicant{3}{11}{blue}
        \implicant[4pt]{13}{11}{red}
      \end{Karnaugh}
    \end{center}
  \end{figure}

  Para el segmento c:
  \begin{figure}[H]
    \begin{center}
      \begin{Karnaugh}{X}{Y}{Z}{W}
        \contingut{
          1,
          1,
          ,
          1,
          1,
          1,
          1,
          1,
          1,
          1,
          1,
          1,
          ,
          1,
          X,
          X
        }
        \implicant{0}{5}{black}
        \implicant[4pt]{1}{11}{green}
        \implicant[4pt]{7}{14}{blue}
        \implicant{8}{10}{red}
      \end{Karnaugh}
    \end{center}
  \end{figure}

  Para el segmento d:
  \begin{figure}[H]
    \begin{center}
      \begin{Karnaugh}{X}{Y}{Z}{W}
        \contingut{
          1,
          ,
          1,
          1,
          ,
          1,
          1,
          ,
          1,
          1,
          ,
          1,
          1,
          ,
          X,
          X
        }
        \implicantdaltbaix[4pt]{0}{8}{black}
        \implicantdaltbaix[4pt]{3}{11}{green}
        \implicantcostats{0}{2}{blue}
        \implicant[4pt]{2}{6}{red}
      \end{Karnaugh}
    \end{center}
  \end{figure}

  Para el segmento e:
  \begin{figure}[H]
    \begin{center}
      \begin{Karnaugh}{X}{Y}{Z}{W}
        \contingut{
          1,
          ,
          1,
          ,
          ,
          ,
          1,
          ,
          1,
          ,
          1,
          1,
          1,
          1,
          X,
          X
        }
      \end{Karnaugh}
    \end{center}
  \end{figure}

  Para el segmento f:
  \begin{figure}[H]
    \begin{center}
      \begin{Karnaugh}{X}{Y}{Z}{W}
        \contingut{
        1,
        ,
        ,
        ,
        1,
        1,
        1,
        ,
        1,
        1,
        1,
        1,
        1,
        1,
        X,
        X
        }
      \end{Karnaugh}
    \end{center}
  \end{figure}

  Para el segmento g:
  \begin{figure}[H]
    \begin{center}
      \begin{Karnaugh}{X}{Y}{Z}{W}
        \contingut{
          ,
          ,
          1,
          1,
          1,
          1,
          1,
          ,
          1,
          1,
          1,
          ,
          ,
          1,
          X,
          X
        }
      \end{Karnaugh}
    \end{center}
  \end{figure}
