

Se tiene la siguiente tabla de verdad para las entradas $S_1$ y $S_0$:

\begin{table}[H]
  \centering
  \begin{tabular}{c c c c c c c}
    \toprule
    $S_1$ & $S_0$ & LRV & LAV & LVV & LRP & LVP \\ \cmidrule(r){1-2} \cmidrule(l){3-7}
    0     & 0     & 0   & 0   & 1   & 1   & 0  \\ \cmidrule(r){1-2} \cmidrule(l){3-7}
    0     & 1     & 0   & 1   & 0   & 1   & 0  \\ \cmidrule(r){1-2} \cmidrule(l){3-7}
    1     & 0     & 1   & 0   & 0   & 0   & 1  \\ \cmidrule(r){1-2} \cmidrule(l){3-7}
    1     & 1     & 1   & 0   & 0   & 1   & 0  \\
    \bottomrule
  \end{tabular}
\end{table}

Se resuelven los mapas de Karnaugh para cada salida.

\begin{figure}[H]
  \begin{subfigure}{0.19\textwidth}
    \centering
    \begin{Karnaughquatre}{$S_1$}{$S_0$}
      \minterms{2, 3}
      \implicant{2}{3}{blue}
    \end{Karnaughquatre}
    \caption{LRV}
  \end{subfigure}
  \begin{subfigure}{0.19\textwidth}
    \centering
    \begin{Karnaughquatre}{$S_1$}{$S_0$}
      \minterms{1}
      \implicantsol{1}{blue}
    \end{Karnaughquatre}
    \caption{LAV}
  \end{subfigure}
  \begin{subfigure}{0.19\textwidth}
    \centering
    \begin{Karnaughquatre}{$S_1$}{$S_0$}
      \minterms{0}
      \implicantsol{0}{blue}
    \end{Karnaughquatre}
    \caption{LVV}
  \end{subfigure}
  \begin{subfigure}{0.19\textwidth}
    \centering
    \begin{Karnaughquatre}{$S_1$}{$S_0$}
      \minterms{0, 1, 3}
      \implicant{0}{1}{blue}
      \implicant[3pt]{1}{3}{red}
    \end{Karnaughquatre}
    \caption{LRP}
  \end{subfigure}
  \begin{subfigure}{0.19\textwidth}
    \centering
    \begin{Karnaughquatre}{$S_1$}{$S_0$}
      \minterms{2}
      \implicantsol{2}{blue}
    \end{Karnaughquatre}
    \caption{LVP}
  \end{subfigure}
\end{figure}

\begin{align*}
  LRV &= S_1 &
  LAV &= \overline{S_1} S_0 &
  LVV &= \overline{S_1} \phantom{\cdot} \overline{S_0} &
  LRP &= \overline{S_1} + S_0 &
  LVP &= S_1 \overline{S_0}
\end{align*}

Se implementan estas funciones.
